%%%%%%%%%%%%%%%%%%%%%%%%%%%%%%%%%%%%%%%%%
% Thin Sectioned Essay
% LaTeX Template
% Version 1.0 (3/8/13)
%
% This template has been downloaded from:
% http://www.LaTeXTemplates.com
%
% Original Author:
% Nicolas Diaz (nsdiaz@uc.cl) with extensive modifications by:
% Vel (vel@latextemplates.com)
%
% License:
% CC BY-NC-SA 3.0 (http://creativecommons.org/licenses/by-nc-sa/3.0/)
%
%%%%%%%%%%%%%%%%%%%%%%%%%%%%%%%%%%%%%%%%%

%----------------------------------------------------------------------------------------
%	PACKAGES AND OTHER DOCUMENT CONFIGURATIONS
%----------------------------------------------------------------------------------------

\documentclass[a4paper, 12pt]{article} % Font size (can be 10pt, 11pt or 12pt) and paper size (remove a4paper for US letter paper)

\usepackage[protrusion=true,expansion=true]{microtype} % Better typography
\usepackage{graphicx} % Required for including pictures
\usepackage[utf8]{inputenc}
\usepackage[margin=1.0in]{geometry}
\usepackage{url}
\usepackage{fancyhdr}
\usepackage{amsmath}
\usepackage{setspace}
\usepackage{enumitem}
\usepackage{float}
\setlength\parindent{0pt} % Removes all indentation from paragraphs

\usepackage[T1]{fontenc} % Required for accented characters
\usepackage{times} % Use the Palatino font

\usepackage{listings}
\usepackage{color}
\lstset{mathescape}

\definecolor{dkgreen}{rgb}{0,0.6,0}
\definecolor{gray}{rgb}{0.5,0.5,0.5}
\definecolor{mauve}{rgb}{0.58,0,0.82}

\lstset{frame=tb,
   language=c++,
   aboveskip=3mm,
   belowskip=3mm,
   showstringspaces=false,
   columns=flexible,
   basicstyle={\small\ttfamily},
   numbers=none,
   numberstyle=\tiny\color{gray},
   keywordstyle=\color{blue},
   commentstyle=\color{dkgreen},
   stringstyle=\color{mauve},
   breaklines=true,
   breakatwhitespace=true
   tabsize=3
}
\linespread{1.00} % Change line spacing here, Palatino benefits from a slight increase by default

\makeatletter
\renewcommand{\@listI}{\itemsep=0pt} % Reduce the space between items in the itemize and enumerate environments and the bibliography

\renewcommand\abstractname{Résumé}
\renewcommand\refname{Références}
\renewcommand\contentsname{Table des matières}
\renewcommand{\maketitle}{ % Customize the title - do not edit title and author name here, see the TITLE block below
\begin{center} % Right align

\vspace*{25pt} % Some vertical space between the title and author name
{\LARGE\@title} % Increase the font size of the title

\vspace{125pt} % Some vertical space between the title and author name

{\large\@author} % Author name

\vspace{125pt} % Some vertical space between the author block and abstract
Dans le cadre du cours
\\INF8225 - Techniques probabilistes et d'apprentissage
\vspace{125pt} % Some vertical space between the author block and abstract
\\\@date % Date
\vspace{125pt} % Some vertical space between the author block and abstract

\end{center}
}

%----------------------------------------------------------------------------------------
%	TITLE
%----------------------------------------------------------------------------------------

\title{TP1} 

\author{\textsc{Guillaume Arruda 1635805} % Author
\vspace{10pt}
\\{\textit{École polytechnique de Montréal}}} % Institution

\date{3 Février 2016} % Date

%----------------------------------------------------------------------------------------

\begin{document}

\thispagestyle{empty}
\clearpage\maketitle % Print the title section
\pagebreak[4]
%----------------------------------------------------------------------------------------
%	En tête et pieds de page 
%----------------------------------------------------------------------------------------

\setlength{\headheight}{15.0pt}
\pagestyle{fancy}
\fancyhead[L]{INF8225}
\fancyhead[C]{}
\fancyhead[R]{TP1}
\fancyfoot[C]{\textbf{page \thepage}}

%----------------------------------------------------------------------------------------
%	ESSAY BODY
%----------------------------------------------------------------------------------------
\section*{Question 1}
La partie PMTK3 de cette question est dans le fichier question1.m de l'archive.
\subsection*{Explaining away}
Dans un réseau bayésien où C est l'enfant de A et B, A et B sont indépendant tant qu'il n'existe pas d'observation sur C.
Une fois que C est observé, un changement au probabilité de A ou B a un influence sur l'autre. Une augmentation de la probabilité de A
diminue la probabilité de B, car il est suffisant pour expliquer l'occurence de C et vice-versa.
\subsection*{Serial blocking}
Dans un réseau bayésien où C est l'enfant de B et B est l'enfant de A, les 3 variables sont dépendantes. Toutefois, si B est observé,
un changement dans la probabilité de A ne peut pas affecter C et vice-versa. L'observation de B coupe la chaine de dépendance entre les variables.
\subsection*{Divergent blocking}
Dans un réseau bayésin où A et C sont les enfants de B, A,B et C sont dépendant. Toutefois, si B est observé, un changement dans la probabilité de A n'affecte pas celle de C et vice-versa. L'observation de B coupe la chaine de dépendance entre les variables.
\section*{Question 2}
L'implémentation de la question 2 est dans le fichier mkDetecterCambriolageDgm.m de l'archive.
\subsection*{c)}
P(Cambriolage = V | MarieAppelle = V, JeanAppelle = F) = 0.008337\\
P(Cambriolage = V | MarieAppelle = F, JeanAppelle = V) = 0.044852\\
P(Cambriolage = V | MarieAppelle = V, JeanAppelle = V) = 0.445950\\
P(Cambriolage = V | MarieAppelle = F, JeanAppelle = F) = 0.000110\\
P(Cambriolage = V | MarieAppelle = V) = 0.016284\\
P(Cambriolage = V | MarieAppelle = F) = 0.235300\\
\subsection*{d)} 
P(Cambriolage = V) = 0.001000\\
P(Tremblement = V) = 0.002000\\
P(Alarme = V) = 0.002516\\
P(MarieAppelle = V) = 0.052139\\
P(JeanAppelle = V) = 0.001994\\
\subsection*{e)}
\subsubsection*{i)}
\begin{equation*}
P(J=V) = \sum_{ctam}{P(C,T,A,J=V,M)}
\end{equation*}
\\
\begin{equation*}
P(J=V) = \sum_{c} (P(C) \cdot \sum_{t} (P(T) \cdot \sum_{a} (P(A|C,T) \cdot P(J=V|A,T) \cdot \sum_{m} P(M|A))))
\end{equation*}
\subsubsection*{ii)}
\begin{equation*}
P(C=V|J=V) = \frac{P(C=V,J=V)}{P(J=V)}
\end{equation*}
\\
\begin{equation*}
P(C=V|J=V) = \frac{P(C=V) \cdot \sum_{t} (P(T) \cdot \sum_{a} (P(A|C=V,T) \cdot P(J=V|A,T) \cdot \sum_{m} P(M|A)))}{P(J=V)}
\end{equation*}
\section*{Question 3}
%----------------------------------------------------------------------------------------
\end{document}
\grid
\grid
